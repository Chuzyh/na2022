\documentclass[12]{article}%12pt即为*四号字
\usepackage{ctex}%引入中文包
\usepackage{graphicx}%插入图片的包
\usepackage{geometry}%设置A4纸页边距的包
\usepackage{url}
\geometry{left=3.18cm,right=3.18cm,top=2.54cm,bottom=2.54cm}%设置页边距
\linespread{1}%设置行间距



\begin{document}
\begin{center}
    \LARGE\songti\textbf{Chapter 1 homework} \\%标题
    \large\kaishu\textbf{褚朱钇恒\qquad 3200104144}%一般是我的姓名
\end{center}
    \section{Theoretical questions}
        \subsection{I}
            The width of the interval at the nth step is $2^{-n+1}$, the max possible distance between r and midpoint is $1$.
        \subsection{II}
            At the nth step, the width of interval is $\frac{b_0-a_0}{2^n}$, the max absolute error is $\frac{b_0-a_0}{2^{n+1}}$, so the max relative error is $\frac{\frac{b_0-a_0}{2^{n+1}}}{a_0}$

            As the relative error is no greater than $\epsilon$, so $\frac{\frac{b_0-a_0}{2^{n+1}}}{a_0} \le \epsilon$.

            Then we have $log_2(b_0-a_0)-n-1-log(a_0)\le log(\epsilon)$.

            So if $n \ge \frac{log(b_0-a_0)-log(\epsilon)-log(a_0)}{log(2)}-1$, the relative error is no greater than $\epsilon$.

        \subsection{III}
            $p^{'}(x)=12x^2-4x$
            
            The result is in the following table.
            \begin{table}[htbp]
                \centering
                \small
                \begin{tabular}{|c|c|c|c|c|}
                \hline
                \textbf{$n$} & \textbf{$x_n$} & \textbf{$p(x_n)$} & \textbf{$p'(x_n)$} & \textbf{$x_n-\frac{p(x_n)}{p'(x_n)}$} \\ \hline
                \textbf{0}   & -1.0000      & -3.0000           & 16.0000            & -0.8125                               \\ \hline
                \textbf{1}   & -0.8125      & -0.4658           & 11.1719            & -0.7708                               \\ \hline
                \textbf{2}   & -0.7708      & -0.0201           & 10.2129            & -0.7688                               \\ \hline
                \textbf{3}   & -0.7688      & -3.9801           & 10.1686            & -0.7688                               \\ \hline
                \textbf{4}   & -0.7688      &                   &                    &                                       \\ \hline
                \end{tabular}
            \end{table}
        \subsection{IV}
            Assume $\alpha$ is the true root, then from Taylor's expansion we have
            $$
f(x_n)=f(\alpha)+f^{'}(\xi)(x_n-\alpha)=f^{'}(\xi)(x_n-\alpha),\xi\in [\alpha,x_n]
            $$

            And combine with $x_{n+1}=x_n-\frac{f(x_n)}{f^{'}(x_0)}$, there is 

            $$
x_{n+1}=x_n-\frac{f^{'}(\xi)(x_n-\alpha)}{f^{'}(x_0)}
            $$

            Subtract $\alpha$ on both side, we get
            $$
            |x_{n+1}-\alpha|=|x_n-\alpha-\frac{f^{'}(\xi)(x_n-\alpha)}{f^{'}(x_0)}|
            $$

            This is equivalent to $e_{n+1}=|1-\frac{f^{'}(\xi)}{f^{'}(x_0)}|e_n$

            Then we have $C=|1-\frac{f^{'}(\xi)}{f^{'}(x_0)}|$ and $s=1$ meet the question's requirement.

        \subsection{V}
            
\end{document}