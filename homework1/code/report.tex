\documentclass[12]{article}%12pt即为*四号字
\usepackage{ctex}%引入中文包
\usepackage{graphicx}%插入图片的包
\usepackage{geometry}%设置A4纸页边距的包
\usepackage{url}
\geometry{left=3.18cm,right=3.18cm,top=2.54cm,bottom=2.54cm}%设置页边距
\linespread{1}%设置行间距



\begin{document}
\begin{center}
    \LARGE\songti\textbf{Chapter 1 Programming Assignments} \\%标题
    \large\kaishu\textbf{褚朱钇恒\qquad 3200104144}%一般是我的姓名
\end{center}
    \section{Programming Assignments}
        \subsection{A}
            The abstract base class and three method is in EquationSolver.h.
        
        \subsection{B}
            \subsubsection{$x^{-1}-\tan(x)=0$}
                $root=0.860333589\ f(root)=0.000000000$
            \subsubsection{$\frac{1}{x}-2^x=0$}
                $root=0.641185745\ f(root)=0.000000000$
            \subsubsection{$2^{-x}+e^x+2cos(x)-6=0$}
                $root=1.829383602\ f(root)=0.000000000$
            \subsubsection{$\frac{x^3+4x^2+3x+5}{2x^3-9x^2+18x-2}=0$}
                $root=0.117876567\ f(root)=6092072270288133.000000000$
                
                This root is wrong, because bisection method found the root of the denominator.
        
        \subsection{C}
            $x-tan(x)=0$

            $root_1=4.493409458 \ f(root_1)=-0.000000000$

            $root_2=7.725251837 \ f(root_2)=0.000000000$
        \subsection{D}
            \subsubsection{$sin(\frac{x}{2}-1=0$}
                $root=3.141592628\ f(root)=0.000000000$
            \subsubsection{$\frac{1}{x}-2^x=0$}
                $root=1.306326940\ f(root)=0.000000000$
            \subsubsection{$2^{-x}+e^x+2cos(x)-6=0$}
                $root=-0.188685403\ f(root)=6092072270288133.000000000$
        \subsection{E}
            Three methods' got the same result $0.166166035$. 
        \subsection{F}
            \subsubsection{a}
                 $\alpha=32.972174822^\circ$
            \subsubsection{b}
                 $\alpha=33.168903820^\circ$
            \subsubsection{c}
                The program got $\alpha=528.5^\circ$, because the original equation has infinite solutions in the real number field.
\end{document}