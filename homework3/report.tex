\documentclass[12]{article}%12pt即为*四号字
\usepackage{ctex}%引入中文包
\usepackage{graphicx}%插入图片的包
\usepackage{geometry}%设置A4纸页边距的包
\usepackage{url}
\usepackage{amsthm,amsmath,amssymb}
\usepackage{mathrsfs}
\geometry{left=3.18cm,right=3.18cm,top=2.54cm,bottom=2.54cm}%设置页边距
\linespread{1}%设置行间距



\begin{document}
\begin{center}
    \LARGE\songti\textbf{Chapter 3 homework} \\%标题
    \large\kaishu\textbf{褚朱钇恒\qquad 3200104144}%一般是我的姓名
\end{center}
    \section{Theoretical questions}
        \subsection{I}
            $s(1)=1,s^{'}(1)=3,s^{''}(1)=6$

            插值可得$p(x)=7x^3-18x^2+12x$,故$s^{''}(0)=-36\neq 0$

            故不是自然样条
        \subsection{II}
            \subsubsection{a}
                在每个区间上,f有三个待定系数,故共有$3(n-1)$个待定系数。

                在每个中间节点上,有$f_{i-1}=f_i,f^{'}_{i-1}=f^{'}_i$,引入两个条件。

                在每个形值点上,有$f_i=f(x_i)$,引入$n$个条件。

                故还需要确定$3(n-1)-2(n-2)-n=1$个条件。

            \subsubsection{b}
                在$x_i$处做泰勒展开得$p_i(x)=f_i+m_i(x-x_i)+a_i(x-x_i)^2$,将$p_i(x_{i+1})=f_{i+1}$,得

                $p_i(x)=f_i+m_i(x-x_i)+\frac{f_{i+1}-f_i-m_i(x_{i+1}-x_i)}{(x_{i+1}-x_i)^2}(x-x_i)^2$
            
            \subsubsection{c}
                根据(b)得,$m_{i+1}=-m_i+2\frac{f_{i+1}-f_i}{x_{i+1}-x_i}$,故可以递推求得$m_2,\cdots,m_{n-1}$

        \subsection{III}
            $s(0)=1+c,s^{'}(0)=3c,s^{''}(0)=6c$,故$s_2(x)=1+c+3cx+3cx^2+ax^3$

            由s为自然样条,$s^{''}(1)=6c+6a=0$,故$a=-c$,即$s_2(x)=1+c+3cx+3cx^2-cx^3$

            $s(1)=-1 \Rightarrow c=-\frac{1}{3}$

        \subsection{IV}
            \subsubsection{a}
                设$s_1(x)=a_1x^3+bx^2+cx+1$,$s_2(x)=a_2x^3+bx^2+cx+1$,

                由$f(-1)=f(1)=0,s^{''}(-1)=s^{''}(1)=0$,解得

                $s_1(x)=-\frac{1}{2}x^3-\frac{3}{2}x^2+1$
                ,$s_2(x)=\frac{1}{2}x^3-\frac{3}{2}x^2+1$
                
            \subsubsection{b}
                $\int_{-1}^{1}[s^{''}(x)]^2dx=6$

                (i)

                $g(x)=-x^2+1$

                $\int_{-1}^1[g^{''}(x)]^2dx=8 > \int_{-1}^{1}[s^{''}(x)]^2dx$

                (ii)

                $\int_{-1}^1[f^{''}(x)]^2dx=\frac{\pi^4}{16} \approx 6.08 > \int_{-1}^{1}[s^{''}(x)]^2dx$

        \subsection{V}
            \subsubsection{a}
            
\end{document} 