\documentclass[12]{article}%12pt即为*四号字
\usepackage{ctex}%引入中文包
\usepackage{graphicx}%插入图片的包
\usepackage{geometry}%设置A4纸页边距的包
\usepackage{url}
\usepackage{amsthm,amsmath,amssymb}
\usepackage{mathrsfs}
\geometry{left=3.18cm,right=3.18cm,top=2.54cm,bottom=2.54cm}%设置页边距
\linespread{1}%设置行间距



\begin{document}
\begin{center}
    \LARGE\songti\textbf{Chapter 2 homework} \\%标题
    \large\kaishu\textbf{褚朱钇恒\qquad 3200104144}%一般是我的姓名
\end{center}
    \section{Theoretical questions}
        \subsection{I}
            $x_0=1,f_0=1,x_1=2,f_2=\frac{1}{2} \Rightarrow p_1(f;x)=-\frac{1}{x}+\frac{3}{2}$

            So, $f(x)-p_1(f;x)=\frac{1}{x}+\frac{x}{2}-\frac{3}{2}=\frac{1}{\xi^3(x)}(x-1)(x-2)$

            $\Rightarrow \xi(x)=\sqrt[3]{\frac{1}{2x}}$

            So,$\max \xi(x)=\sqrt[3]{\frac{1}{2}},\min \xi(x)=\sqrt[3]{\frac{1}{4}},\max f^{''}(\xi(x))=\max 4x=8$

        \subsection{II}
            First, find an interpolation polynomial $p(x)$ of degree $n$ that satisfies $p(x_i)=\sqrt{f_i},i=0,1,\dots,n$

            Then, let $p_2(x)=p^2(x)$, then, we can verify than $p_2(x)\ge0$ , $p_2(x_i)=f_i$ and $p_2\in \mathbb{P}^+_{2n}$
        
        \subsection{III}
            $\forall t$, $f[t]=f(t)=e^t=\frac{(e-1)^0}{0!}e^t$

            Assume than when $n=1,2,\dots,k$,$forall t\in \mathbb{R}, f[t,t+1,\dots,t+n]=\frac{(e-1)^n}{n!}e^t$

            Then $\forall t\in\mathbb{R}, f[t,t+1,\dots,t+k+1]=\frac{f[t+1,t+2,\dots,t+n+1]-f[t,t+1,\dots,t+n]}{t+1-t}=\frac{(e-1)^{n+1}}{(n+1)!}e^t$

            By induction, the original proposition is proved.

            $f[0,1,\dots,n]=\frac{(e-1)^n}{n!}e^0=\frac{(e-1)^n}{n!}=\frac{1}{n!}f^{(n)}(\xi)$

            $\Rightarrow (e-1)^n=e^{\xi}$

            So, $\xi=nlog(e-1)$ and it is located at the right side of $\frac{n}{2}$
        
        \subsection{IV}
        \begin{tabular}{c|c|c|c|c}
            0 & 5  \\
            1 & 3  & -2 \\
            3 & 5  & 1 & 1 \\
            4 & 12 & 7 & 2 & 0.25
        \end{tabular}

            So, $p_3=5-2x+(x-1)x+0.25x(x-1)(x-3)$

            $p_3^{'}(x)=\frac{3x^2-9}{2}\Rightarrow x_{\min}=\sqrt{3}$

        \subsection{V}
        \begin{tabular}{c|c|c|c|c|c|c}
            0 & 0 \\
            1 & 1   & 1 \\
            1 & 1   & 7   & 6 \\
            1 & 1   & 7   & 21  & 15\\
            2 & 128 & 127 & 120 & 99 & 42 \\
            2 & 128 & 448 & 321 & 201 & 102 & 30
        \end{tabular}

        $f[0,1,1,1,2,2]=30=\frac{f^{(5)(\xi)}}{5!}=\frac{7*6*5*4*3x^2}{120}$

        $\Rightarrow \xi=\sqrt{\frac{10}{7}}$
        \subsection{VI}
        \begin{tabular}{c|c|c|c|c|c}
            0 & 1 \\
            1 & 2 & 1 \\
            1 & 2 & -1 & -2 \\
            3 & 0 & -1 & 0   & $\frac{2}{3}$ \\
            3 & 0 & 0  & 0.5 & 0.25 & -$\frac{5}{36}$
        \end{tabular}

        $p(x)=1+x-2(x-1)x+\frac{2}{3}(x-1)^2x-\frac{5}{36}(x-3)(x-1)^2x$

        $f(2)\approx p(2)=\frac{11}{18}$

        $|f(2)-p(2)|\le \frac{|f^{(5)}(\xi|}{5!}(2-1)^2(2-3)^2(2)\le \frac{M}{60}$

        \subsection{VII}
            $\forall x\in \mathbb{R},\Delta^0f(x)=0!h^0f[x_0]$

            Assume that $\forall x\in \mathbb{R},\Delta^kf(x)=k!h^kf[x_0,\dots,x_k]$

            Then $\Delta^{k+1}f(x)=\Delta^kf(x+h)-\Delta^kf(x)=k!h^k(f[x_1,\dots,x_{k+1}]-f[x_0,\dots,x_{k}])=(k+1)!h^{k+1}f[x_0,\dots,x_{k+1}]$

            The second equation can prove similarly.

        \subsection{VIII}
            $$\frac{\partial}{x_0}f[x_0,\dots,x_n]=\lim_{h \rightarrow 0}\frac{f[x_0+h,x_1,\dots,x_n]-f[x_0,x_1,\dots,x_n]}{h}$$

            $$=\lim_{h \rightarrow 0}f[x_0,x_0+h,x_1,\dots,x_n]=f[x_0,x_0,x_1,\dots,x_n]$$

            Similarly

            $$\frac{\partial}{x_k}f[x_0,\dots,x_n]=\lim_{h \rightarrow 0}f[x_0,x_k+h,x_1,\dots,x_n]=f[x_k,x_0,x_1,\dots,x_n]$$

        \subsection{IX}
            When $x\in[-1,1],a_0=1$, $\min\max_{x\in[-1,1]}|a_0x^n+\dots+a_n|=\frac{1}{2^{n-1}}$

            Let $x=\frac{b-a}{2}(y+1),y\in[-1,1]$

            then $p(x)=q(y)=a_0(\frac{b-a}{2}y+\frac{a+b}{2})^n+\dots+a_n$

            $\Rightarrow\min\max_{y\in[-1,1]}|\frac{2^n}{a_0(b-a)^n}q(y|=\frac{1}{2^{n-1}}$

            $\Rightarrow\min\max_{x\in[a,b]}|p(x)|=\frac{a_0(b-a)^n}{2^{2n-1}}$

        \subsection{X}
            $T_n(x)=cos(n \arccos x)$

            Assume $\exists p_0 \in \mathbb{P}^a_n,\Vert \hat{p}_n\Vert_\infty > \Vert p_0 \Vert_\infty$, then $\max_{x\in[-1,1]}|p_0|<\frac{1}{T_n(a)}$
            
            
            Let $Q(x)=\hat{p}(x)-p_0(x)$,then $Q(x^{'}_k)=\frac{(-1)^k}{T_n(a)}-p_0(x^{'}_k)$, then $Q(x)$ has alternating signs at these $n + 1$ points.

            However, the degree of $Q(x)$ is no larger than $n$, so $Q(x)\equiv 0$, which is a contradiction to assumption. So the original proposition is true.
        

        \subsection{XI}
            $$\sum_{k=0}^nb_{n,k}(t)=((1-t)+t)^n=1$$

            $$(p+q)^n=\sum_{k=0}^n\tbinom{n}{k}p^kq^{n-k}$$
            $$\Rightarrow n(p+q)^{n-1}=\sum_{k=1}^nk\tbinom{n}{k}p^{k-1}q^{n-k}$$
            $$\Rightarrow np(p+q)^{n-1}=\sum_{k=1}^nk\tbinom{n}{k}p^{k}q^{n-k}$$
            Let $p=t,q=1-t$, we have
            $$\sum_{k=0}^nkb_{n,k}(t)=nt$$


            $$(p+q)^n=\sum_{k=0}^n\tbinom{n}{k}p^kq^{n-k}$$
            $$\Rightarrow np(p+q)^{n-1}=\sum_{k=1}^nk\tbinom{n}{k}p^{k}q^{n-k}$$
            $$\Rightarrow n(n-1)p(p+q)^{n-2}+n(p+q)^{n-1}=\sum_{k=1}^nk^2\tbinom{n}{k-1}p^{k}q^{n-k}$$
            $$\Rightarrow n(n-1)p^2(p+q)^{n-2}+np(p+q)^{n-1}=\sum_{k=1}^nk^2\tbinom{n}{k}p^{k}q^{n-k}$$
            Let $p=t,q=1-t$, we have
            $$\sum_{k=0}^nk^2b_{n,k}(t)=n(n-1)t^2+nt$$

            then 
            $$\sum_{k=0}^n(k-nt)^2b_{n,k}(t)=\sum_{k=0}^n(k^2-2knt+n^2t^2)b_{n,k}(t)$$

            $$=n(n-1)t^2+nt-2n^2t^2+n^2t^2=nt-nt^2=nt(1-t)$$
\end{document}